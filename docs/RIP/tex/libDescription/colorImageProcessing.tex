\section{Color Image Processing}
An RGB color image is MxNx3 array of color pixels, where each color pixel is triplet corresponding to the red, green and blue components of an RGB image at a specific spatial location.
\newline
RGB images are stored as uint8 array and their range values is [0,255]
\newline
Colors are generally represented using RGB values. However, there are other color spaces whose use in some applications may be more convenient and/or appropriate. These include NTSC, YCbCr, HSV, CMY, CMYK and his color spaces.

For {\bf color images} the transformation model is in the form:
\begin{equation}
	s_i = T_i(r_i) i = 1,2,...,n
\end{equation}
where ri and si are the color components of the input and output images, n is the dimension of the color space of ri and Ti are referred to as full-color transformation (or mapping) function.

For {\bf monochrome images} the transformation model is in the form
\begin{equation}
	s_i = T_i(r)  i = 1,2,...,n
\end{equation}
where r denotes gray-level values, si and Ti are as above, and n is the number of color components in si.

rgb2Y
Y2rgb
other color spaces?
RGB2Y 

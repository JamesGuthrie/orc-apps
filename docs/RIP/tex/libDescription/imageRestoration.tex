\section{Image Restoration}

The objective of restoration is to improve a given image in some predefined sense. Restoration attempts to reconstruct or recover an image that has been degraded by using a priori knowledge of the degradation phenomenon. Thus, restoration techniques are oriented toward modeling the degradation and applying the inverse process in order to recover the original image.

\subsection{Noise Modeling}
The ability to simulate the behavior and effects of noise is central to image restoration. To basic type of noise model can be used: noise in the spacial domain and noise in the frequency domain.

\subsubsection{Spatial Noise Modeling}
Spacial noise model is described by the noise probability density function.

\subsubsection{Frequency Noise Modeling}
Frequency noise model is described by various Fourier properties of the noise.

\subsection{Filtering Restoration}

\subsubsection{Spatial Filtering Restoration}
The  image degradation process is modeled as a degradation function that with an additive noise term operate on an input image f(x,y) to produce a degraded image image g(x,y):
\begin{equation}
	g(x,y) = H[f(x,y)] + n(x,y)
\end{equation}

\subsubsection{Frequency Filtering Restoration}
Supposing that H is a linear, spacial invariant process, it is possible to model the degradation function in the frequency domain as:
\begin{equation}
	G(u,v) = H(u,v)F(u,v) + N(u,v)
\end{equation}
where the terms in capital letters are the Fourier transform of the corresponding terms in the convolution equation for the equation presented in the Spatial Filtering Restoration.

{\bf Restoration in the presence of noise only-spatial filtering
adaptive spatial filters
periodic noise reduction – frequency domain filtering
wiener filtering
constrained latest square (regularized) filtering
lucky-richardson algorithm}


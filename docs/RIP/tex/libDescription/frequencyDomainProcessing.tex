\section{Frequency Domain Processing}

Methods in this category are based on filtering carried out in the frequency domain via the Fourier Transform and are denoted by the expression:
\begin{equation}
	G(u,v) = T[F(u,v)]
\end{equation}
where $F(x,y)$ is the $2D$ Fourier Transformation of the input image $f(x,y)$, $G(x,y)$ the output processed image in the frequency domain and $T$ is the operator of $F$, defined over a specified neighborhood about point $(x,y)$.

\subsection{Linear Frequency Filtering}

As for the linear spatial filtering, The mechanism of linear frequency filtering consist simply in moving the center of a filter mask (or filter window) from point to point in the frequency transformation $F$ of the image $f$.

\subsection{Non Linear Frequency Filtering}

The mechanism of non linear frequency filtering is the same as the non linear spatial filtering, soft that all the considerations are made from the frequency transformation $F$ of the image $f$.

{\bf Blurring: Averaging / Lowpass Filtering 
Sharpening: Differencing / Highpass Filtering 
Gaussian Highpass / Lowpass Filter}


\section{Spatial Domain Processing}
The term spatial refer to the image plane itself. Methods in this category are based on direct manipulation of pixels in an image and are denoted by the expression:
\begin{equation}
	g(x,y) = T[f(x,y)]
\end{equation}
where $f(x,y)$ is the input image, $g(x,y)$ the output processed image and T is the operator of $f$, defined over a specified neighborhood about point $(x,y)$.

\subsection{Linear Spatial Filtering}

The mechanism of linear spatial filtering consist simply in moving the center of a filter mask (or filter window) from point to point in an image f. At each point (x,y), the response of the filter at that point is the sum of products of the filter coefficients and the corresponding neighborhood pixels in the area spanned by the filter mask. 

\subsection{Non Linear Spatial Filtering}

The mechanism of non linear spatial filtering is the same as discussed for the linear one. However, whereas linear spatial filtering is based on computing the sum of products (witch is a linear operation), non linear spatial filtering is based on nonlinear operations involving the pixel in the neighborhood.
\newline

A special case for the non linear spacial filtering is the intensity  or gray-level transformation function. In this case the $T$ transformation has a neighborhood of size $1x1$ (a single pixel): the value of $g$ at $(x,y)$ depends only on the intensity of $f$ at that point.
